\section{\LaTeX 基本用法}
本节介绍\LaTeX 的基本用法。关于\LaTeX 更详细的介绍,读者可以自行阅读\LaTeX~的相关教材,如刘海洋老师的《LaTeX入门》\cite{Liu2013}。

\subsection{数学公式}
数学公式易学难精,如果比较考究,需要注意的事项非常多。下面的例子里面,上面或者左边是在\LaTeX~文档中输入的代码,下面或者右边是编译之后输出的内容\footnote{这样可以得到一个脚注。}。

\subsubsection{基本用法}
数学公式分行内公式和行间公式。下面的例子说明了如何使用行内公式和行间公式,其中,上半部分是LaTeX中的内容,下半部分是输出。


将字母置于两个美元符号之间,即可得到一个行内公式,下面是一个简单的行内公式举例:
\begin{myexample2}{行内公式}
这是一个行内公式$a^2+b^2=c^2$。
\end{myexample2}

如果公式两边各有两个美元符号,可以得到一个不带编号的行间公式,需要注意的是行间公式应该看成句子的一部分,因此大部分情况需要在行间公式添加标点符号。下面是一个简单的行间公式的例子:
\begin{myexample}{不带编号的行间公式}
这里是一个行间公式
$$a^2+b^2=c^2,$$
其中$a$和$b$是直角边的长度,$c$是斜边的长度。
\end{myexample}

下面介绍一些常用的公式书写方式:
\begin{enumerate}
\item 这是上标下标
$$a_1,\ a_2,\ \ldots,\ a_n, \ x^2,\ x^{n+1},\ \mathrm{e}^{i\pi} + 1 = 0;$$

\item 这是求和,积分,求极限
$$\sum_{i=1}^n i = \frac{n(n+1)}{2}, \
\int_0^1 x^2\, dx = \frac{1}{3},\
\lim_{x \to 0} \frac{\sin x}{x} = 1;$$

\item  这是特殊符号
$$\infty,\ \partial,\ \nabla,\ \alpha,\ \beta,\ \gamma,\ \ldots;$$

\item 这是分数
$$\frac{a}{b+c},\ \frac{a}{b+\frac{c}{d}};$$

\item 这是矩阵、行列式
$$
A = \begin{bmatrix}
1 & 2 & 3 \\
4 & 5 & 6 \\
7 & 8 & 9
\end{bmatrix},\ 
B = \begin{pmatrix}
1 & 2 & 3 \\
4 & 5 & 6 \\
7 & 8 & 9
\end{pmatrix},\ 
C = \begin{vmatrix}
1 & 2 & 3 \\
4 & 5 & 6 \\
7 & 8 & 9
\end{vmatrix},\ 
D = \begin{Vmatrix}
1 & 2 & 3 \\
4 & 5 & 6 \\
7 & 8 & 9
\end{Vmatrix}.
$$

\end{enumerate}


如果想得到一个带编号的行间公式,可以使用公式环境,下面是一个带编号的行间公式:
\begin{myexample}{带编号的行间公式}
这里是一个带编号的行间公式
\begin{equation}\label{eq:exm1}
    a^2+b^2=c^2.
\end{equation}
\end{myexample}

公式环境中的\verb|\label|命令后的参数,可以让我们方便地进行公式引用,如果不需要引用,则不必添加label。。
你可以使用合适的字母为公式命名,并使用\verb|\eqref{}|进行引用。下面是一个公式引用的例子:
\begin{myexample}{公式引用}
在这里,我们可以使用\eqref{eq:exm1}引用上述公式。
\end{myexample}

在LaTeX中,章节、图、表、定理等的引用与公式的引用类似,只需在相应位置加上\verb|\label{标签}| 即可引用,需要注意的是,其余引用,使用\verb|\ref{标签}|即可。注意,公式引用使用的是\verb|\eqref{}|,它可以在编号两侧自动加上括号。


如果公式太长,可以使用split环境进行换行,下面是一个公式换行的例子:
\begin{myexample2}{公式换行}
\begin{equation}
    \begin{split}
        a+b+c &= c+d+e \\
              &+ f+g+h.
    \end{split}
\end{equation}
\end{myexample2}
\noindent 注意,在上面的例子中,我们使用\&让等号和加号对齐。

如果需要使用方程组,可以使用以下几种方式得到:
\begin{myexample2}{使用cases环境}
\begin{equation}
    \begin{cases}
        x + y &= 1, \\ % & 用来对齐
        x - y &= 2.
    \end{cases}
\end{equation}
\end{myexample2}

\begin{myexample2}{使用aligned环境}
\begin{equation}
\left\{
    \begin{aligned}
        x + y &= 1, \\ %& 用来对齐
        x - y &= 2.
    \end{aligned}
\right.
\end{equation}
\end{myexample2}


\begin{myexample2}{使用align环境}
\begin{align}
    x + y &= 1, \\ %& 用来对齐
    x - y &= 2.
\end{align} 
\end{myexample2}

\begin{myexample2}{使用eqnarray环境}
\begin{eqnarray}
    x + y &= 1, \\ %& 用来对齐
    x - y &= 2.
\end{eqnarray} 
\end{myexample2}

如果需要带大括号且每个方程都有编号的公式,则需要使用其它宏包,如empheq,的支持。
\begin{myexample}{使用其它宏包}
\begin{empheq}[left=\empheqlbrace]{align}
    x + y &= 1, \\
    x - y &= 2, \\
    2x + 3y &= 5.
\end{empheq}
\end{myexample}



\subsubsection{特别注意事项}
值得注意的是,在数学公式环境中,有些符号应该使用直体,例如:
\begin{enumerate}
	\item 微分符号$d$应该用直体,写作$\d x$;
	\item $max$应写作$\max$, $min$应写作$\min$;
	\item 三角函数应写作直体,如$sin$应写作$\sin$;
	\item 英文缩写等应写成直体,如$\mathrm{s.t.}$;
    \item 公式中包含的单词,应写成直体,如$\mathrm{for}$。
\end{enumerate}
在公式环境中,使用{\ttfamily $\backslash$mathrm\{\}}可以获得直体字母,示例如下:
\begin{myexample}{}
$a^2+b^2=c^2,\ \mathrm{for}\ a>0$. %可以用\ 表示空格
\end{myexample}

\subsection{图形和表格}\label{sec:tabfig}

我们可以使用figure环境和table环境插入图片和表格。

\begin{listonly}{图形示例}
\begin{figure}[htbp]
    \begin{center}
        \includegraphics[width=0.35\linewidth]{lixinlogo.jpg}\quad
        \includegraphics[width=0.35\linewidth]{lixinlogo.jpg}
    \end{center}
    \caption[图形示例]{两个校徽}\label{fig:exm1}
\end{figure}
\noindent 可以使用图~\ref{fig:exm1}引用该图片。% \noindent 表示该行不缩进
\end{listonly}

\noindent 上述内容可以得到如下输出:
\begin{figure}[htbp]
	\begin{center}
		\includegraphics[width=0.35\linewidth]{lixinlogo.png}\quad
		\includegraphics[width=0.35\linewidth]{lixinlogo.png}\\
            \includegraphics[width=0.35\linewidth]{lixinlogo.png}\quad
		\includegraphics[width=0.35\linewidth]{lixinlogo.png}
	\end{center}
	\caption{四个校徽}\label{fig:exm1}
\end{figure}

\noindent 可以使用图~\ref{fig:exm1}引用该图片。% \noindent 表示改行不缩进


如果想让一个图中的每个小图都有小标题,则可以使用subfig宏包中的subfloat环境,下面是一个这种例子。
\begin{figure}[htbp]    
  \centering            
  \subfloat[第一个子图标题]   
  {\includegraphics[width=0.35\linewidth]{lixinlogo.png} \label{fig:subfig1}}\quad
  \subfloat[第二个子图标题]
  {\includegraphics[width=0.35\linewidth]{lixinlogo.png} \label{fig:subfig2}}\\
  \subfloat[第三个子图标题]   
  {\includegraphics[width=0.35\linewidth]{lixinlogo.png} \label{fig:subfig3}}\quad
  \subfloat[第四个子图标题]
  {\includegraphics[width=0.35\linewidth]{lixinlogo.png} \label{fig:subfig4}}\\
  \caption{四个校徽}    
  \label{fig:subfig}
\end{figure}

\noindent 子图也可以进行交叉引用,例如图~\ref{fig:subfig1}。


表格也是呈现数据或者数值结果的一个重要方式。在\LaTeX~中,我们一般使用table环境和tabular环境产生表格,下面是一个表格和表格引用的例子。
\begin{listonly}{表格示例}
\begin{table}[htbp]
    \caption{三线表}\label{tab:exp1}
    \begin{center}
        \newcolumntype{C}{>{\centering\arraybackslash}X}%
        \begin{tabularx}{0.9\linewidth}{CCCCCC}
            \toprule
            1 & 2 & 3 & 4 & 5 & 6\\
            \midrule
            1 & 2 & 3 & 4 & 5 & 6\\
            1 & 2 & 3 & 4 & 5 & 6\\
            1 & 2 & 3 & 4 & 5 & 6\\
            \bottomrule
        \end{tabularx}
    \end{center}
\end{table}
\noindent 可以使用表~\ref{tab:exm1}引用该表格。% \noindent 表示改行不缩进
\end{listonly}

\noindent 上述内容可以得到如下输出:


\begin{table}[htbp]
	\caption{三线表}\label{tab:exm1}
	\begin{center}
		\newcolumntype{C}{>{\centering\arraybackslash}X}%
		\begin{tabularx}{0.9\linewidth}{CCCCCC}
			\toprule
			1 & 2 & 3 & 4 & 5 & 6\\
			\midrule
			1 & 2 & 3 & 4 & 5 & 6\\
			1 & 2 & 3 & 4 & 5 & 6\\
			1 & 2 & 3 & 4 & 5 & 6\\
			\bottomrule
		\end{tabularx}
	\end{center}
\end{table}
\noindent 可以使用表~\ref{tab:exm1}引用该表格。% \noindent 表示改行不缩进

\subsection{代码抄录}
使用lstlisting环境可以进行代码抄录,本模板中的设置适用于Matlab, 用户也可以定制适合自己所用语言的设置。


\begin{myexample}{代码抄录环境示例}
\begin{lstlisting}[caption=代码抄录示例,language=Matlab,label=code:exm1]
function s = f(x)
% This Program gives the area of a circle.
% S = F(x)
    pi = 3.14;
    
    s = pi*x^2;
end
\end{lstlisting}
可以使用代码~\ref{code:exm1}引用代码。
\end{myexample}

\subsection{参考文献}

参考文献需要再bibref.bib中进行编写,如为图书,应像下方一样输入:
\begin{mybox}
\begin{lstlisting}
@book{Liu2013,
	title={LATEX入门},
	author={刘海洋},
	publisher={电子工业出版社},
	address = {北京},
	year={2013},
}
\end{lstlisting}	
\end{mybox}

\noindent 如为文章,则应像下方一样输入:
\begin{mybox}
\begin{lstlisting}
@article{Driscoll2024aaa,
	title={{AAA} rational approximation on a continuum},
	author={Driscoll, Tobin A and Nakatsukasa, Yuji and Trefethen, Lloyd N},
	journal={SIAM Journal on Scientific Computing},
	volume={46},
	number={2},
	pages={A929--A952},
	year={2024},
	publisher={SIAM}
}
\end{lstlisting}
\end{mybox}
注意上面标题中的AAA三个字母用大括号括起来了,如果不这样做,输出的参考文献会显示为Aaa.

\noindent 引用时,使用{\ttfamily$\backslash$cite\{\}}即可引用,例如:
\begin{myexample}{参考文献使用示例}
参考文献可以这样引用\cite{Liu2013}。
两篇或者多篇可以这样引用\cite{Liu2013,Driscoll2024aaa}。
\end{myexample}

\noindent 其它文献类型如下
\begin{mybox}
\begin{tabular}{p{4.5cm}p{2cm}p{6.5cm}}
文献类型         &  标识代码 &  Entry Type\\
\hline
普通图书         &  M        &  @book \\
图书的析出文献   &  M        &  @incollection\\
会议录           &  C        &  @proceedings\\
会议录的析出文献 &  C        &  @inproceedings 或 @conference\\
汇编             &  G        & \textcolor{red}{@collection}\\
报纸             &  N        & \textcolor{red}{@newspaper}\\
期刊的析出文献   &  J        &  @article\\
学位论文         &  D        &  @mastersthesis 或 @phdthesis\\
报告             &  R        &  @techreport\\
标准             &  S        &  \textcolor{red}{@standard}\\
专利             &  P        &  \textcolor{red}{@patent}\\
数据库           &  DB       &  \textcolor{red}{@database}\\
计算机程序       &  CP       &  \textcolor{red}{@software}\\
电子公告         &  EB       & \textcolor{red}{@online}\\
档案             &  A        & \textcolor{red}{@archive}\\
舆图             &  CM       & \textcolor{red}{@map}\\
数据集           &  DS       & \textcolor{red}{@dataset}\\
其他             &  Z        &  @misc\\
\hline
\end{tabular}

\vspace{0.5cm}
{\bfseries 注:}红色字体的类型不是 BibTeX 的标准文献类型。
\end{mybox}

英文图书这样引用\cite{book:fan2014}, 译注这样引用\cite{book:xie2012},中文图书这样引用\cite{book:xu2010},中文期刊这样引用\cite{art:yuan2012},英文期刊这样引用\cite{art:frese2013},论文集这样引用\cite{inpro:jia2011},报告这样引用\cite{tech:zh2013},博士学位论文这样引用\cite{phd:ma2011},硕士学位论文这样引用\cite{master:ma2011},专利可以这样引用\cite{pat:deng2006},技术标准可以这样引用\cite{satand:gbt25100},新闻可以这样引用\cite{news:yu2013},数据库可以这样引用\cite{DB:zhao2014},电子期刊可以这样引用\cite{JOL:2008},电子公告可以这样引用\cite{Online:2013}。


\noindent 一般情况下,在编辑器中插入需要的文献类型,就会将参考文献所需填写的项目列出;或者也可以从百度学术或者谷歌学术导出,直接复制到bibref.bib中即可。其它关于参考文献的信息,可以参考\url{https://github.com/zepinglee/gbt7714-bibtex-style}。