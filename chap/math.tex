\section{数学公式和图表}
数学公式在数学写作上应该已经学过,这里不做介绍,如有需要,可以参考文献~\cite{Liu2013}。

\subsection{数学公式}
值得注意的是,微分符号$d$应该用直体,写作$\d x$,$max$应写作$\max$, $min$应写作$\min$, $sin$应写作$\sin$。

\begin{definition}[勾股定理]
	假设直角三角形的三边长为$a$、 $b$、 $c$,斜边长为$c$,则
	$$a^2+b^2=c^2.$$
\end{definition}

\begin{lemma}[勾股定理]
	假设直角三角形的三边长为$a$、 $b$、 $c$,斜边长为$c$,则
	$$a^2+b^2=c^2.$$
\end{lemma}
\begin{theorem}[勾股定理]
	假设直角三角形的三边长为$a$、 $b$、 $c$,斜边长为$c$,则
	$$a^2+b^2=c^2.$$
\end{theorem}




\subsection{图表}\label{sec:tabfig}

\subsubsection{图形}
\begin{figure}[htbp]
	\begin{center}
		\includegraphics[width=0.5\linewidth]{lixinlogo.jpg}
	\end{center}
	\caption[图形示例]{校徽}
\end{figure}

\subsubsection{表格}
\begin{table}[htbp]
	\caption{三线表}\label{tab:dem}
	\begin{center}
		\newcolumntype{C}{>{\centering\arraybackslash}X}%
		\begin{tabularx}{0.9\linewidth}{CCCCCC}
			\toprule
			1 & 2 & 3 & 4 & 5 & 6\\
			\midrule
			1 & 2 & 3 & 4 & 5 & 6\\
			1 & 2 & 3 & 4 & 5 & 6\\
			1 & 2 & 3 & 4 & 5 & 6\\
			\bottomrule
		\end{tabularx}
	\end{center}
\end{table}


\begin{myexample}
 $f(x)$
\end{myexample}

\begin{lstlisting}[caption=main]
function xdot = f(x, t)
	r = 0.25; k = 1.4;
    a = 1.5; b = 0.16; c = 0.9; d = 0.8;

  	xdot(1) = r*x(1)*(1 - x(1)/k) - a*x(1)*x(2)/(1 + b*x(1));
  	xdot(2) = c*a*x(1)*x(2)/(1 + b*x(1)) - d*x(2);

end
\end{lstlisting}
